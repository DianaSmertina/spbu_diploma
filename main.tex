
% ОБЯЗАТЕЛЬНО ИМЕННО ТАКОЙ documentclass!
% (Основной кегль = 14pt, поэтому необходим extsizes)
% Формат, разумеется, А4
% article потому что стандарт не подразумевает разделов
% Глава = section, Параграф = subsection
% (понятия "глава" и "параграф" из документа, описывающего диплом)
\documentclass[a4paper,article,14pt]{extarticle}

% Подключаем главный пакет со всем необходимым
\usepackage{spbudiploma_tempora}

% Пакеты по желанию (самые распространенные)
% Хитрые мат. символы
\usepackage{euscript}
% Таблицы
\usepackage{longtable}
\usepackage{makecell}
% Картинки (можно встявлять даже pdf)
\usepackage[pdftex]{graphicx}

\usepackage{amsthm,amssymb, amsmath}
% remove leading space of mod
\usepackage{textcomp}


\newcommand{\Mod}[1]{\ \mathrm{mod}\ #1}

\begin{document}

% Титульник в файле titlepage.tex
% --------------------- Стандарт СПбГУ для ВКР --------------------------
% Автор: Тоскин Николай, itonik@me.com
% Если заметили ошибку, напишите на email
% Если хотите добавить изменение самостоятельно, GitHub: . PR-s welcome!
% Использованы материалы:
% habr.com/ru/post/144648/
% cpsconf.ru
% Текст:
% http://edu.spbu.ru/images/data/normativ_acts/local/20181030_10432_1.pdf
% Титульный лист:
% http://edu.spbu.ru/images/data/normativ_acts/local/20180703_6616_1.pdf
% -----------------------------------------------------------------------

% Титульный лист диплома СПбГУ
% Временное удаление foot на titlepage
\newgeometry{left=30mm, top=20mm, right=15mm, bottom=20mm, nohead, nofoot}
\begin{titlepage}
\begin{center}
% Первый символ съедается, первым знаком поставлен Ы
\textbf{Санкт--Петербургский}
\textbf{государственный университет}

\vspace{35mm}

\textbf{\textit{\large ПОТТЕР Гарри Джеймсович}} \\[8mm]
% Название
\textbf{\large Выпускная квалификационная работа}\\[3mm]
\textbf{\textit{\large Оптимизация заклятий в распределенной сети
мозгошмыг}}

\vspace{20mm}
% Булшит
Уровень образования: бакалавриат\\
Направление 01.03.02 «Прикладная математика и информатика»\\
Основная образовательная программа СВ.5005.2015
«Прикладная математика, фундаментальная информатика и программирование»\\
Профиль «Исследование и проектирование систем управления\\ и обработки сигналов»\\[30mm]


% Научный руководитель, рецензент
% Сходить в уч отдел и узнать, правильно ли
\begin{flushright}
{Научный руководитель:} \\
профессор, кафедра компьютерных технологий \\ и систем, д.ф. - м.н.  Веремей~Евгений Игоревич
\end{flushright}
\begin{flushright}
{Рецензент:} \\
профессор, кафедра компьютерных технологий \\и систем, д.ф. - м.н.  Веремей~Евгений Игоревич
\end{flushright}

\vfill 

{Санкт-Петербург}
\par{2019 г.}
\end{center}
\end{titlepage}
% Возвращаем настройки geometry обратно (то, что объявлено в преамбуле)
\restoregeometry
% Добавляем 1 к счетчику страниц ПОСЛЕ titlepage, чтобы исключить 
% влияние titlepage environment
\addtocounter{page}{1}

% Содержание
\tableofcontents
\pagebreak

\specialsection{Введение}

С развитием современных медиа и интернета увеличивается объем передаваемых данных. Вместе с этим растет потребность в безопасности 
данных, которые представляют собой некоторую ценность. Традиционные методы защиты информации представляет криптография. Чаще всего 
информация защищается с помощью секретного алгоритма или ключа. Но у такого подхода есть проблемы: если злоумышленник перехватит 
ключ или скомпрометирует одну из сторон, то он легко получит доступ к секрету.

В 1979 году A. Shamir представил (ссылка) алгоритм 
разделения секрета, который позволяет разбить секрет на $n$ долей таким образом, что знание $K$ и более долей позволяет восстановить 
секрет, а знание $K-1$ и менее долей делает восстановление секрета невозможным. В последние десятилетия было предложено множество 
алгоритмов разделения секрета для электронных изображений. В данной работе будет рассмотрен и дополнен алгоритм обратимого 
разделения секрета, реализована библиотека для использования в веб-приложениях и пример минимального проекта, использующего эту 
библиотеку

\newpage
\specialsection{Цель и постановка задачи}

Целью данной работы является написание библиотеки для языка JavaScript, для разделения секретного цветного электронного изображения,
с долями, не подобными шуму. Для достижения этой цели были поставлены следующие задачи:
\begin{enumerate}
    \item Исследование предметной области
    \item Выбор алгоритма
    \item Модификация алгоритма для соответствия поставленным требованиям
    \item Написание библиотеки
    \item Написание минимального веб-приложения, позволяющего продемонстрировать работу программы
    \item Тестирование библиотеки и сравнение с имплементациями на других языках 
\end{enumerate}

\newpage
\specialsection{Обзор литературы}

В рамках спецификации современных стандартов, базовые сценарии поведения пользователей призваны к ответу. Банальные, но неопровержимые выводы, а также представители современных социальных резервов формируют глобальную экономическую сеть и при этом - представлены в исключительно положительном свете.

Есть над чем задуматься: предприниматели в сети интернет будут описаны максимально подробно. Приятно, граждане, наблюдать, как сторонники тоталитаризма в науке заблокированы в рамках своих собственных рациональных ограничений. Есть над чем задуматься: некоторые особенности внутренней политики объявлены нарушающими общечеловеческие нормы этики и морали. Как принято считать, тщательные исследования конкурентов смешаны с неуникальными данными до степени совершенной неузнаваемости, из-за чего возрастает их статус бесполезности.

Лишь предприниматели в сети интернет, которые представляют собой яркий пример континентально-европейского типа политической культуры, будут преданы социально-демократической анафеме. Есть над чем задуматься: стремящиеся вытеснить традиционное производство, нанотехнологии являются только методом политического участия и ограничены исключительно образом мышления! Разнообразный и богатый опыт говорит нам, что постоянный количественный рост и сфера нашей активности напрямую зависит от новых предложений.

\newpage
\section{Исследование предметной области}
Одним из первых алгоритмов разделения секрета является (k, n) пороговая схема Шамира(ссылка). В ее основе лежит интерполяция 
полиномов. Пусть $D$ -- некоторая секретная информация, представленная в форме числа. Выберем простое число $p: p > D, p > N$.
Чтобы разделить секрет на $n$ частей возьмем случайный полином степени $k-1$ 
\begin{equation}
    q(x) = a_0 + a_1 x +...+ a_{k-1} x^{k-1},
    a_0=D, a_i<p
\end{equation}
и вычислим
\begin{equation}
    D_1=q(1)\Mod{p}, ..., D_i=q(i)\Mod{p}, ..., D_n=q(n)\Mod{p}
\end{equation}
Число $p$ будет публичным для всех участников, числа $D_i$ назовем долями. Участника схемы, который хочет разделить секрет и 
формирует доли назовем дилером.

Имея $k$ и более долей можно восстановить секрет $D$ при помощи полиномиальной интерполяции.
%добавить инфу про восстановление 
Допустим, злоумышленнику удалось получить доступ к $k-1$ долям, тогда для каждого $D': 0<D'<p$ он может восстановить единственный полином степени $k-1$, такой, что $q_0=D'$ и 
$q_i=D_i$. Так как $a_i$ случайны, эти $p$ полиномов с одинаковой вероятностью являются искомыми, злоумышленник не получает никакой
информации о секрете. 

Схема Шамира позволяет разделить секрет, представленный в форме числа и используется в основном для защиты ключей. Изображение так же 
можно представить в форме числа, но при обычном размере изображения (для примера 256х256) и значениях пикселя (0-255)х3 для rgb изображений
будет тратиться огромное количество памяти. Поэтому на основе схемы Шамира были разработаны алгоритмы разделения секрета для изображений. 
Их можно разделить на три категории - схемы визуальной криптографии(VCS), полиномиальные схемы и схемы, основанные на Китайской теоремы об остатках. 

В 1994 году Moni Naor и Adi Shamir (ссылка) представили первую VCS, на ее основе были разработаны другие модификации.
В VCS схемах доли обычно печатаются на прозрачных носителях и восстанавливается путем наложения частей друг на друга. 
Основным преимуществом таких схем является отсутствие необходимости вычислений при восстановлении секрета.
Примечательной для применения в веб-разработке является VCS схема WEB-VC (ссылка). Алгоритм восстановления секрета основан 
на возможности установить прозрачность элемента в таблице каскадных стилей (CSS). Основными недостатками таких схем является 
наличие помех в восстановленном секретном изображении и возможность использования только с бинарными изображениями.

Полиномиальные схемы используются чаще из-за лучшего качества восстановленного секрета и в общем случае 
не требуют увеличения количества пикселей. Но у них есть и недостатки -- относительно высокая вычислительная сложность 
восстановления секрета $O(k*log^2(k))$ для каждого пикселя и небольшие потери в качестве восстановленного секретного изображения.
% Дополнить про 1 достойную схему


Во многих схемах дилер отправляет участникам шумо-подобные доли. Введём понятие изображений для прикрытия -- это произвольные 
изображения, использующиеся для генерации долей. Сгенерированные доли 
являются изображениями в оттенках серого, похожими на изображения прикрытия. Использование изображений прикрытия 
вместо шумо-подобных долей снижает риск привлечения внимания к долям злоумышленников, улучшает возможности 
по их менеджменту. 

В данной работе будет рассматриваться алгоритм Reversible Image Secret Sharing (ссылка). Он основан на китайской 
теореме об остатках. В качестве секретной картинки выступает изображение в оттенках серого $(0-255)$.


\newpage
\section{Описание алгоритма}
Начнем описание работы алгоритма с формулировки Китайской теоремы об остатках.

Если $ a_1,...,a_n \in N $ попарно взаимно просты, то для 
$$\forall r_1,...,r_n \in N : 0\leq r_i<a_i, \forall i\in \overline{1,n}$$
найдется $N: N \Mod a_i = p_i, \forall i\in \overline{1,n}$ 

Эта теорема позволяет за линейное время решать систему линейных модулярных уравнений следующего типа:
\begin{gather}
    y \equiv a_1 \Mod m_1 \\
    y \equiv a_2 \Mod m_2 \\
    ... \\
    y \equiv a_k \Mod m_k
\end{gather}
Алгоритм решения (ссылка):
\begin{enumerate}
    \item Вычисляем $M=\prod\limits_{i = 1}^k m_i$
    \item $\forall i\in \overline{1,k}$ вычисляем $M_i={{M} \over {m_i}}$
    \item С помощью расширенного алгоритма Евклида $\forall i\in \overline{1,k}$ находим ${M_i}^{-1}$ обратное по модулю для $M_i$
    \item Получаем $y \equiv \sum\limits_{i=1}^k a_i M_i {M_i}^{-1} \Mod M$
\end{enumerate}

Предложенный алгоритм состоит из двух частей: формирование долей и восстановление секрета. Опишем их более подробно.

\subsection{Формирование долей}
Описание входных данных:

\begin{itemize}
    \item Секретное изображение $S$ размера $W_S * H_S$ пикселей в оттенках серого (значения пикселей 0-255)
    \item $n$ - количество долей
    \item $k$ - минимальное количество долей для восстановления секрета
    \item $n$ изображений $C_i$ размера $W_S * H_S$ - бинарные (значения пикселей 0-1) изображения прикрытия для каждого из участников
\end{itemize}

Описание выходных данных:

\begin{itemize}
    \item $n$ изображений $SC_i$ размера $W_S * H_S$ - сгенерированные доли
    \item $m_i$ - приватное число для каждой доли
    \item $p, T$ - публичные для всех участников числа для восстановления секрета
\end{itemize}

Алгоритм:
\begin{enumerate}
    \item Выберем число $p$ и $n$ взаимно простых чисел $m_i$ таких, что 
    $$128 \leq p < m_i \leq 256, \text{НОД}(m_i, p)=1, \forall i \in \overline{1,n}$$ 
    \item Вычислим $M=\prod\limits_{i = 1}^k m_i$, $N=\prod\limits_{i = 1}^{k-1} m_{n-i+1}$
    \item Если $M<pN$ перейдем к шагу 1
    \item Вычислим $T=\left[ {{\lfloor{{M}\over{p}}-1\rfloor}\over{2}} \right]$
    \item Для каждого секретного пикселя $x$ с координатами $[w, h]$ повторяем шаги 6-7
    \item Если $0 \leq x < p$, выберем случайное $ A \in [T+1, {\lfloor{{M} \over {p}} - 1\rfloor}]$ и вычислим
    $y = x + Ap$. 
    
    Если $x \geq p$ выберем случайное $ A \in [0, T)$ и вычислим $y = x - p + Ap$
    \item Если выполняется одно из следующих условий, то вычисляем $a_i = y \Mod p$, устанавливаем $SC_i=a_i$ и
    переходим к следующему пикселю, иначе возвращаемся на шаг 6.
    \begin{gather}
        SC_i[w,h] \geq TH_{i1}, \text{ если } C_i[w,h] = 1 \\
        SC_i[w,h] \leq TH_{i0}, \text{ если } C_i[w,h] = 0
    \end{gather}
\end{enumerate}

\subsection{Восстановление секрета}

Описание входных данных:
\begin{itemize}
    \item $n$ долей $SC_i$ ($n \geq k$) и соответствующие им $m_i$
    \item Публичные числа $T, p$
\end{itemize}

Описаные выходных данных:
\begin{itemize}
    \item Восстановленный секрет $S'$
    \item Восстановленные изображения прикрытия $C_{i}'$ размера $[W_S, H_S]$
\end{itemize}

Алгоритм
\begin{enumerate}
    \item Восстановливаем изображения прикрытия с помощью бинаризации. Для каждого пикселя $C_{i}'[w, h]$ устанавливаем значение 
    $$\text{Если } SC_{i}[w, h]>{{m_i}\over{2}} \text{, то } 1 \text{, иначе } 0$$
    \item Для каждой позиции пикселя $[w, h]$, $a_i = SC_i[w, h]$, с помощью описанного выше алгоритма (ссылка) решаем систему 
    линейных уравнений по модулю:
    \begin{gather}
        y \equiv a_1 \Mod m_1 \\
        ... \\
        y \equiv a_i \Mod m_i \\
        ... \\
        y \equiv a_n \Mod m_n
    \end{gather}
    \item Вычисляем $T^{*}=\lfloor{{y} \over {p}}\rfloor$. Если $T^{*}\leq T$, то $x = y \Mod p$, иначе $x = (y \Mod p) + p$.
    
    $S'[w, h] = x$
\end{enumerate}

\subsection{}
ремарка про коэф TH


Описанный алгоритм отлично подходит для цели работы, за исключением цвета картинки. Поэтому
было принято решение расширить исходный алгоритм для использования с цветными секретными картинками. 
Это было достигнуто с помощью увеличения количества пикселей в картинках прикрытия и кодирования каждого канала цвета 
в определенном пикселе доли.

формулировка улучшения для цветных картинок

более подробный анализ плюсов и минусов полученного алгоритма по сравнению с интерполяционными

рассказ про библиотеку и как я ее офигенно загрузил на нпм и какая она в открытом доступе
пару слов про приложение со скринами




Нумерованная формула:

\begin{equation}
    i^2 = -1.
    \label{eq:my_ref}
\end{equation}

Тест ссылки на формулу \ref{eq:my_ref}.


\specialsection{Выводы}
Жизнь --- тлен.
\pagebreak

\specialsection{Заключение}

С другой стороны, консультация с широким активом обеспечивает актуальность форм воздействия. Следует отметить, что выбранный нами инновационный путь создает необходимость включения в производственный план целого ряда внеочередных мероприятий с учетом комплекса благоприятных перспектив. В частности, реализация намеченных плановых заданий влечет за собой процесс внедрения и модернизации поэтапного и последовательного развития общества. В частности, новая модель организационной деятельности способствует подготовке и реализации стандартных подходов и тому подобных экспериментов.

% Библиография в cpsconf стиле
% Аргумент {1} ниже включает переопределенный стиль с выравниванием слева
\begin{thebibliography}{1}
\bibitem{voc} Griffin D.W., Lim J.S. \flqq Multiband excitation vocoder\frqq. IEEE ASSP-36 (8), 1988, pp. 1223-1235.
\bibitem{vo2} Griffin D.W., Lim J.S. \flqq Multiband excitation vocoder\frqq. IEEE ASSP-36 (8), 1988, pp. 1223-1235.
\end{thebibliography}
\end{document}